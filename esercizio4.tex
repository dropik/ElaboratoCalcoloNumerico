\paragraph{Esercizio 4.} Per calcolare la funzione seno, utilizzando solo operazioni algebriche elementari, si può sviluppare la funzione in serie di Taylor, cioè
\[
sin(x) = x - \frac{x^3}{3!} + \frac{x^5}{5!} - \frac{x^7}{7!} + \ldots
\]
In questo modo possiamo costruire un algoritmo, cui implementazione è riportata nel file \emph{+esercizio4/seno.m}. Per confrontare il risultato dell'algoritmo con la funzione \emph{sin} di Matlab, è stato eseguito su un insieme $X$ contenente 500 punti equidistanti sull'intervallo $[-\pi; \pi]$, e l'errore medio assoluto ottenuto era
\[
MAE = 9.9017e-17
\]