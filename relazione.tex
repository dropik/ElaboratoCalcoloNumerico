\documentclass[12pt, a4paper]{article}
\usepackage[utf8]{inputenc}
\usepackage{amsmath}
\usepackage[document]{ragged2e}

\title{Elaborato per il corso Calcolo Numerico}
\author{Daniil Ryzkov}
\date{Anno Accademico 2020-2021}

\begin{document}

\maketitle
\newpage

\paragraph{Esercizio 1.} Sviluppando in serie di Taylor abbiamo che

\[ f(x-h) \simeq f(x) - hf'(x) + \frac{h^2}{2}f''(x) - \frac{h^3}{6}f'''(x) \]
\[ f(x-2h) \simeq f(x) - 2hf'(x) + 2h^2f''(x) - \frac{4h^3}{3}f'''(x) \]

Che possiamo sostituire nell'espressione iniziale ed ottenere

\[
\begin{split}
& \frac{3f(x) - 4\left[f(x) - hf'(x) + \frac{h^2}{2}f''(x) - \frac{h^3}{6}f'''(x)\right] + f(x) - 2hf'(x) + 2h^2f''(x) - \frac{4h^3}{3}f'''(x)}{2h} =  \\
& = \frac{2hf'(x) - \frac{2h^3}{3}f'''(x)}{2h} = f'(x) - \frac{h^2}{3}f'''(x)
\end{split}
\]

ovvero

\[ \frac{3f(x) - 4f(x-h) + f(x-2h)}{2h} = f'(x) + O(h^2) \]

\end{document}