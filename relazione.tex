\documentclass[12pt, a4paper]{article}
\usepackage[utf8]{inputenc}
\usepackage{amsmath}
\usepackage[document]{ragged2e}
\usepackage{listings}
\usepackage{xcolor}

\definecolor{background}{rgb}{0.97,0.97,0.97}
\definecolor{frame}{rgb}{0.7,0.7,0.7}

\lstdefinestyle{mStyle}{
  backgroundcolor=\color{background},
  frame=single,
  rulecolor=\color{frame}
}

\lstset{style=mStyle}

\title{Elaborato per il corso Calcolo Numerico}
\author{Daniil Ryzkov}
\date{Anno Accademico 2020-2021}

\begin{document}

\maketitle
\newpage

\paragraph{Esercizio 1.} Sviluppando in serie di Taylor abbiamo che

\[ f(x-h) \simeq f(x) - hf'(x) + \frac{h^2}{2}f''(x) - \frac{h^3}{6}f'''(x) \]
\[ f(x-2h) \simeq f(x) - 2hf'(x) + 2h^2f''(x) - \frac{4h^3}{3}f'''(x) \]

Che possiamo sostituire nell'espressione iniziale ed ottenere

\[
\begin{split}
& \frac{3f(x) - 4\left[f(x) - hf'(x) + \frac{h^2}{2}f''(x) - \frac{h^3}{6}f'''(x)\right] + f(x) - 2hf'(x) + 2h^2f''(x) - \frac{4h^3}{3}f'''(x)}{2h} =  \\
& = \frac{2hf'(x) - \frac{2h^3}{3}f'''(x)}{2h} = f'(x) - \frac{h^2}{3}f'''(x)
\end{split}
\]

ovvero

\[ \frac{3f(x) - 4f(x-h) + f(x-2h)}{2h} = f'(x) + O(h^2) \]

\paragraph{Esercizio 2.} Partiamo con valore iniziale pari ad $1$:

\begin{lstlisting}[language=Matlab]
e = 1;
\end{lstlisting}

Dopodiché eseguiamo un ciclo finché la somma $e+1$ non sia pari ad $1$, diminuendo il valore di $e$ a metà ad ogni passo.

\begin{lstlisting}[language=Matlab]
while e + 1 > 1
  e = e / 2;
end
\end{lstlisting}

Questo non è un ciclo infinito, siccome ad un certo punto il valore di $e$ diventerà così piccolo, che sommandolo con $1$ per l'errore di round-off si otterrà esattamente $1$, ed il ciclo si finirà.

Il valore ottenuto dopo questa procedura è pari ad $1.110223e-16$, come si aspettava per lo standard IEEE754, in cui sono disponibili 53 bit per la mantissa. La variabile \verb|eps| di Matlab invece è pari al doppio di questa cifra, siccome la variabile \verb|eps| rappresenta piuttosto la distanza fra $1$ ed il prossimo valore più grande di $1$ rappresentabile con doppia precisione.

\end{document}