\paragraph{Esercizio 10.} Il codice della function \verb|mialu(lu, p, b)| è riportato nel file \emph{src/mialu.m}.

Per quanto riguarda il testing, a parte dei unit test, che sono disponibili nei file \emph{test/plu\_tests.m} e \emph{test/mialu\_tests.m}, possiamo verificare la correttezza delle function scritte negli ultimi due esercizi, risolvendo un sistema generato casualmente. Generiamo la matrice \emph{A} ed il vettore degli ignoti \emph{x} usando la function \verb|rand| di Matlab. Motiplicando \emph{A} per \emph{x} otteniamo il termine noto \emph{b}. Usando la matrice \emph{A} ed il vettore \emph{b} generati, proviamo a risolvere questo sistema tramite le function, che abbiamo appena scritto, e confrontiamo il risultato con il vettore \emph{x} originale.

\textbf{Esempio 1}
\[
A = \begin{pmatrix}
0.9649 & 0.9572 & 0.1419\\
0.1576 & 0.4854 & 0.4218\\
0.9706 & 0.8003 & 0.9157
\end{pmatrix}, x = \begin{pmatrix}
0.7922\\
0.9595\\
0.6557
\end{pmatrix}, b = \begin{pmatrix}
1.7758\\
0.8671\\
2.1373
\end{pmatrix}
\]

Il risultato ottenuto è

\[
\hat{x} = \begin{pmatrix}
7.922073295595542e-01\\
9.594924263929030e-01\\
6.557406991565869e-01
\end{pmatrix}
\]

\[
E_{ass} = 1.1102e-16
\]

\textbf{Esempio 2}
\[
A = \begin{pmatrix}
0.4456 & 0.6797 & 0.9597 & 0.2551 & 0.5472\\
0.6463 & 0.6551 & 0.3404 & 0.5060 & 0.1386\\
0.7094 & 0.1626 & 0.5853 & 0.6991 & 0.1493\\
0.7547 & 0.1190 & 0.2238 & 0.8909 & 0.2575\\
0.2760 & 0.4984 & 0.7513 & 0.9593 & 0.8407
\end{pmatrix}, x = \begin{pmatrix}
0.2543\\
0.8143\\
0.2435\\
0.9293\\
0.3500
\end{pmatrix}, b = \begin{pmatrix}
1.3291\\
1.2994\\
1.1572\\
1.2613\\
1.8446
\end{pmatrix}
\]

Il risultato ottenuto è
\[
\hat{x} = \begin{pmatrix}
2.542821789715319e-01\\
8.142848260688160e-01\\
2.435249687249892e-01\\
9.292636231872273e-01\\
3.499837659848094e-01
\end{pmatrix}
\]

\[
E_{ass} = 4.6629e-16
\]
