\paragraph{Esercizio 11.} Il codice della function \verb|mialdl(a, b)| ed anche di tutte le sue function private sono riportati nella cartella \emph{src/@mialdl}.

Per quanto riguarda il testing, a parte dei unit test, che sono disponibili nel file \emph{test/mialdl\_tests.m}, possiamo verificare la correttezza della function, risolvendo un sistema generato casualmente. Perciò occorre generare una matrice sdp $A$.

La moltiplicazione di una matrice non singolare per la trasposta di questa matrice risulta in una matrice sdp. Cioè, sia $\hat{A}$ una matrice non singolare. La matrice definita come $A = \hat{A} \hat{A}^T$ è una matrice sdp.

\textbf{Dim.}

Dimostriamo prima che la matrice risultante è simmetrica. Infatti,
\[
A^T = (\hat{A}\hat{A}^T)^T = (\hat{A}^T)^T\hat{A}^T = \hat{A}\hat{A}^T = A
\]

Allora $A = A^T$ e la matrice è simmetrica.

Ora fissiamo un $x \in \mathbb{R}^n, x \neq 0$.
\[
x^T A x = x^T \hat{A} \hat{A}^T x = (x^T \hat{A}) (\hat{A}^T x)
\]

Sia $y = \hat{A}^T x$. Allora si ha che
\[
y^T = (\hat{A}^Tx)^T = x^T(\hat{A}^T)^T = x^T\hat{A}
\]

che è la prima parte dell'espressione. Quindi abbiamo che
\[
x^TAx = y^T y = \sum_{i=1}^{n} y_{i}^2 > 0
\]
Cvd.

Sapendo questo possiamo generare una matrice ausiliaria usando la function \verb|rand| di Matlab. Moltiplicando questa matrice non singolare per la sua trasposta otteniamo la matrice $A$. In modo analogo all'esercizio precedente generiamo il vettore $b$. Si hanno quindi due esempi seguenti.

\textbf{Esempio 1}
\[
A = \begin{pmatrix}
0.5920 & 0.7038 & 0.3009\\
0.7038 & 1.0201 & 0.7158\\
0.3009 & 0.7158 & 0.9976
\end{pmatrix}, x = \begin{pmatrix}
0.7513\\
0.2551\\
0.5060
\end{pmatrix}, b = \begin{pmatrix}
0.7765\\
1.1511\\
0.9134
\end{pmatrix}
\]

Il risultato ottenuto è

\[
\hat{x} = \begin{pmatrix}
7.512670593056523e-01\\
2.550951154592698e-01\\
5.059570516651422e-01
\end{pmatrix}
\]

\[
E_{ass} = 4.8110e-16
\]

\textbf{Esempio 2}
\[
A = \begin{pmatrix}
0.6951 & 0.9194 & 0.8561 & 0.9510\\
0.9194 & 1.5177 & 1.1407 & 1.4908\\
0.8561 & 1.1407 & 1.1089 & 1.1224\\
0.9510 & 1.4908 & 1.1224 & 2.2493
\end{pmatrix}, x = \begin{pmatrix}
0.4733\\
0.3517\\
0.8308\\
0.5853
\end{pmatrix}, b = \begin{pmatrix}
1.9201\\
2.7891\\
2.3846\\
3.2233
\end{pmatrix}
\]

Il risultato ottenuto è
\[
\hat{x} = \begin{pmatrix}
4.732888489027274e-01\\
3.516595070629998e-01\\
8.308286278962896e-01\\
5.852640911527238e-01

\end{pmatrix}
\]

\[
E_{ass} = 1.6515e-15
\]
