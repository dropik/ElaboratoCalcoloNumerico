\paragraph{Esercizio 14.} Il codice della function \verb|miaqr(qr, b)| è riportato nel file \emph{src/miaqr.m}.

Per verificare le funciton \verb|qrfat| e \verb|miaqr|, le abbiamo applicati a due sistemi generati usando la function \verb|rand|, confrontando il risultato con quello ottenuto dall'operatore backslash \verb|\| di Matlab. Consideriamo che l'errore assoluto in questo caso è stato calcolato come differenza fra i due risultati ottenuti.

\textbf{Esempio 1}
\[
A = \begin{pmatrix}
0.8147 & 0.0975 & 0.1576\\
0.9058 & 0.2785 & 0.9706\\
0.1270 & 0.5469 & 0.9572\\
0.9134 & 0.9575 & 0.4854\\
0.6324 & 0.9649 & 0.8003
\end{pmatrix}, b = \begin{pmatrix}
0.1419\\
0.4218\\
0.9157\\
0.7922\\
0.9595
\end{pmatrix}
\]

\[
E_{ass} = 2.0354e-16
\]

\textbf{Esempio 2}
\[
A = \begin{pmatrix}
0.6557 & 0.3922 & 0.0971\\
0.0357 & 0.6555 & 0.8235\\
0.8491 & 0.1712 & 0.6948\\
0.9340 & 0.7060 & 0.3171\\
0.6787 & 0.0318 & 0.9502\\
0.7577 & 0.2769 & 0.0344\\
0.7431 & 0.0462 & 0.4387
\end{pmatrix}, b = \begin{pmatrix}
0.3816\\
0.7655\\
0.7952\\
0.1869\\
0.4898\\
0.4456\\
0.6463
\end{pmatrix}
\]

\[
E_{ass} = 2.2204e-16
\]