\paragraph{Esercizi 16-17.} 
\begin{itemize}
  \item $\Phi_{in}(x_j) = \delta_{ij}$:

Supponiamo prima che $j = i$, allora
\[
\Phi_{in}(x_i) = L^2_{in}(x_i)\Big[1 - 2(x_i - x_i)L'_{in}(x_i)\Big] = 1^2\Big[1 - 2 \cdot 0 \cdot L'_{in}(x_i)\Big] = 1
\]

Se invece $j \neq i$, allora $L^2_{in}(x_j) = 0$ e quindi $\Phi_{in}(x_j) = 0$. In questo modo si ha che $\Phi_{in}(x_j) = \delta_{ij}$.

  \item $\Phi'_{in}(x_j) = 0$:

Calcoliamo prima $\Phi'_{in}(x)$
\[
\Phi'_{in}(x) = (L^2_{in}(x))'\Big[1 - 2(x - x_i)L'_{in}(x_i)\Big] + L^2_{in}(x)\Big[1 - 2(x - x_i)L'_{in}(x_i)\Big]'
\]

La derivata di $L^2_{in}(x)$ risulta $2L_{in}(x)L'_{in}(x)$. Poi, sapendo che $L'_{in}(x_i)$ è un valore costante, abbiamo
\[
[1 - 2(x - x_i)L'_{in}(x_i)]' = -2L'_{in}(x_i)
\]

Così possiamo scrivere
\[
\Phi'_{in}(x) = 2L_{in}(x)\Big\{L'_{in}(x)\big[1 - 2(x - x_i)L'_{in}(x_i)\big] - L_{in}(x)L'_{in}(x_i)\Big\}
\]

Se ora fissiamo $j = i$, allora abbiamo che
\[
\Phi'_{in}(x_i) = 2\Big\{L'_{in}(x_i)\big[1 - 2 \cdot 0 \cdot L'_{in}(x_i)\big] - L'_{in}(x_i)\Big\} = 2\Big\{L'_{in}(x_i) - L'_{in}(x_i)\Big\} = 0
\]

Se invece $i \neq j$, allora $L_{in}(x_j) = 0$ e quindi $\Phi'_{in}(x_j) = 0$.

  \item $\Psi_{in}(x_j) = 0$:

Se $j = i$, allora
\[
\Psi_{in}(x_i) = (x_i - x_i)L^2_{in}(x_i) = 0
\]

Se invece $j \neq i$, allora
\[
\Psi_{in}(x_j) = (x_j - x_i) \cdot 0 = 0
\]

  \item $\Psi'_{in}(x_j) = \delta_{ij}$:
  
Calcoliamo prima la derivata
\begin{align*}
\Psi'_{in}(x)& = L^2_{in}(x) + (x - x_i)(L^2_{in}(x))' \\
  &= L^2_{in}(x) + 2(x - x_i)L_{in}(x)L'_{in}(x) \\
  &= L_{in}(x)\Big[L_{in}(x) + 2(x - x_i)L'_{in}(x)\Big]
\end{align*}

Se $j = i$
\[
\Psi'_{in}(x_i) = 1 \cdot \Big[1 + 2 \cdot 0 \cdot L'_{in}(x_i)\Big] = 1
\]

Se invece $j \neq i$, allora $L_{in}(x_j) = 0$ e quindi $\Psi'_{in}(x_j) = 0$. Allora si ha che $\Psi'_{in}(x_j) = \delta_{ij}$.
\end{itemize}

Cvd.