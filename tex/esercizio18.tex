\paragraph{Esercizio 18.} Dimostriamo prima che vale il primo vincolo. Sappiamo già che $\Phi_{in}(x_j) = \delta_{ij}$ e $\Psi_{in}(x_j) = 0$. Allora possiamo valutare il polinomio nei punti $x_j$ come
\[
p(x_j) = \sum_{i = 0}^n f_i \delta_{ij}
\]

Siccome $\delta_{ij} = 1$ solo se $i = j$, allora $p(x_j) = f_j$. E così abbiamo dimostrato il primo vincolo.

Per quanto riguarda il secondo vincolo, calcoliamo prima la derivata del polinomio.
\[
p'(x) = \sum_{i = 0}^n \big[f_i \Phi'_{in}(x) + f'_i \Psi'_{in}(x)\big]
\]

Siccome $\Phi'_{in}(x_j) = 0$ e $\Psi'_{in}(x_j) = \delta_{ij}$, allora possiamo scrivere che
\[
p'(x_j) = \sum_{i = 0}^n f'_i \delta_{ij}
\]

Ed in modo analogo al primo vincolo si ottiene che $p'(x_j) = f'_j$, che è il secondo vincolo. Cvd.