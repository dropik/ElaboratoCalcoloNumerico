\paragraph{Esercizio 2.} Partiamo con valore iniziale pari ad $1$:

\begin{lstlisting}[language=Matlab]
e = 1;
\end{lstlisting}

Dopodiché eseguiamo il ciclo seguente finché la somma $e+1$ non sia pari ad $1$, diminuendo il valore di $e$ a metà ad ogni passo.

\begin{lstlisting}[language=Matlab]
while e + 1 > 1
  e = e / 2;
end
\end{lstlisting}

Questo non è un ciclo infinito, siccome ad un certo punto il valore di $e$ diventerà così piccolo, che sommandolo con $1$ per l'errore di round-off si otterrà esattamente $1$, ed il ciclo si finirà.

Il valore ottenuto dopo questa procedura è pari ad $1.110223e-16$, come si aspettava per lo standard IEEE754, in cui sono disponibili 53 bit per la mantissa in doppia precisione. La variabile \verb|eps| di Matlab invece è pari al doppio di questa cifra, siccome la variabile \verb|eps| rappresenta piuttosto la distanza fra $1$ ed il prossimo valore più grande di $1$ rappresentabile con doppia precisione.

\newpage
