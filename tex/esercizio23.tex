\paragraph{Esercizi 23-24.} Come si può vedere nei risultati, riportati nella tabella \ref{fig:approxRunge}, l'approssimazione più precisa è stata quella tramite spline cubica naturale su 31 ascisse equidistanti. Si vede anche, che la scelta della base del polinomio interpolante non cambia il risultato di approssimazione, in quanto per l'unicità del polinomio hanno i coefficienti uguali.

\begin{figure}
  \centering
  \begin{tabular}{|p{0.5\linewidth}|p{0.3\linewidth}|p{0.2\linewidth}|}
    \hline
    Metodo & Ascisse & $||f - g||$ \\
    \hline\hline
    
    Polinomio interpolante in base di Lagrange & 31 ascisse equidistanti & $37.5900 \cdot 10^4$ \\
    \hline
    Polinomio interpolante in base di Newton & 31 ascisse equidistanti & $37.5900 \cdot 10^4$ \\
    \hline
    Polinomio interpolante in base di Lagrange & 31 ascisse di Chebyshev & $0.1027$ \\
    \hline
    Polinomio interpolante in base di Newton & 31 ascisse di Chebyshev & $0.1027$ \\
    \hline
    Polinomio di Hermite in base di Lagrange & 15 ascisse equidistanti & $1.1278 \cdot 10^3$ \\
    \hline
    Polinomio di Hermite in base di Newton & 15 ascisse equidistanti & $1.1278 \cdot 10^3$ \\
    \hline
    Polinomio di Hermite in base di Lagrange & 15 ascisse Chebyshev & $0.2482$ \\
    \hline
    Polinomio di Hermite in base di Newton & 15 ascisse Chebyshev & $0.2482$ \\
    \hline
    Spline cubica naturale & 31 ascisse equidistanti & $0.0196$ \\
    \hline
    Spline cubica not-a-knot & 31 ascisse equidistanti & $0.0216$ \\
    \hline
  \end{tabular}
  \caption{I risultati dell'approssimazione tramite interpolazione polinomiale e spline della funzione di Runge $f(x) = \frac{1}{1 + x^2}$.}
  \label{fig:approxRunge}
\end{figure}

\begin{figure}
  \centering
  \includegraphics[width=0.9\linewidth]{int_pol_equidist}
  \caption{Polinomio interpolante su 31 ascisse equidistanti.}
\end{figure}

\begin{figure}
  \centering
  \includegraphics[width=0.9\linewidth]{int_pol_chebyshev}
  \caption{Polinomio interpolante su 31 ascisse di Chebyshev.}
\end{figure}

\begin{figure}
  \centering
  \includegraphics[width=0.9\linewidth]{int_hermite_equidist}
  \caption{Polinomio interpolante di Hermite su 15 ascisse equidistanti.}
\end{figure}

\begin{figure}
  \centering
  \includegraphics[width=0.9\linewidth]{int_hermite_chebyshev}
  \caption{Polinomio interpolante di Hermite su 15 ascisse di Chebyshev.}
\end{figure}

\begin{figure}
  \centering
  \includegraphics[width=0.9\linewidth]{spline_cub}
  \caption{Spline cubica naturale su 31 ascisse equidistanti.}
\end{figure}

\begin{figure}
  \centering
  \includegraphics[width=0.9\linewidth]{spline_nak}
  \caption{Spline cubica not-a-knot su 31 ascisse equidistanti.}
\end{figure}