\paragraph{Esercizio 3.} In memoria delle macchine i numeri vengono salvati in base 2. Per questo motivo un numero come 0.1 non può essere rappresentato in una macchina esattamente, siccome per rappresentare $\frac{1}{10}$ in base 2 ci servirebbe un numero infinito dei cifri, che certamente non abbiamo in una macchina, perciò macchine sono costrette a ritagliare tali numeri, portando così gli errori nei calcoli. Lo stesso vale anche per il numero $1.00000000001e+11$, ottenuto come risultato di \verb|a+b|. In questo caso però la macchina deve ritagliare un po' prima la parte frazionaria del numero, siccome deve memorizzare anche $1e+10$ nella mantissa, che naturalmente occupa la sua parte nella memoria. Alla fine, sottraendo \verb|a| dal risultato precedente, la parte $1e+10$ non è più presente nel numero, ma l'errore assoluto ottenuto nel passo precedente rimane lo stesso. Di conseguenza l'errore relativo si aumenta, ed osserviamo il risultato dell'espressione \verb|a+b-a| pari ad $1.000003814697266e-01$ grazie anche al formato di output dei valori numerici che abbiamo impostato a modalità floating point con 15 cifre.